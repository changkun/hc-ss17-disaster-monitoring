\section{Introduction}

In this chapter, we gives an introduction about 
the general backgrounds and challanges of disaster monitoring. Then we briefly introduce 
the related technics that used in the report, 
and put forward our general purspose of this report.

\subsection{Related Works}

\subsubsection{UNICEF Challanges and Satellite Resources}

The United Nations Children's Fund (\textbf{UNICEF\label{idx:unicef}}) \cite{unicef1994state} 
is a United Nations programme headquartered
in New York City that provides humanitarian and developmental assistance to 
children and mothers in developing countries.
It works in 190 countries and territories to protect the rights of every child.
UNICEF has spent 70 years working to improve the lives of children and their families. 
Defending children's rights throughout their lives requires a global presence, 
aiming to produce results and understand their effects. 
For example, in Syrian, the UNICEF works on providing and transporting critical medicine, 
aid and supplies to the refugees living in the war areas \cite{unicef2017report}. 

The challenges UNICEF meets is that there are many hard-to-reach and 
besieged areas (lake of infrastructure), the supplies are extremly difficult to be delivered to these zones 
if the UNICEF is not aware of the real time war situation and the disaster level. 
It leads huge costs for the UNICEF 
which is just a non-profit organization if they would have to hire employees to collect the data of war situation.

With limitations of infrastructure, disaster can only be monitored from the sky level.
Satellite sensors may be used to observe the disaster areas, which mainly gives image information of monitoring areas. 
Zhang et al. \cite{zhang2002flood} discussed the application of a national integreated system using remote sensing, geographic information 
for monitoring and evaluating flood disaster. Their system has been applied for three years, which gives a proof of the success
of satellite sensor informations.

\subsubsection{Human Computation System and Game With A Purpose}


\textbf{Human Computation system \label{idx:hc}} is a paradigm for utilizing the human processing power to solve problems 
that computers cannot yet solve \cite{quinn2011human}.
It is the system of computers and large numbers of humans that work together in order to solve problems 
that could not be solved by either computers or humans alone \cite{quinn2009taxonomy}.
\textbf{Game With A Purpose (GWAP)\label{idx:gwap}} was first proposed in \cite{von2005esp, von2006games}, it
is a human computation technique of outsourcing steps 
within a computational process to humans in an entertaining way (gamification). 
Neverthless, the data collection mechanisms for a game is variety that should be considered in a proper way \cite{von2008designing}.

\subsection{Purpose of the System}

This report sketches a GWAP, aiming at involving citizens to collect data on HtR and BSG \cite{unicef2017report} areas, 
which helps organizations like UNICEF. For the collection mechanism of this game, 
we require game players to select a region upon
the presented satellite images and tag the most relevant tags from the provided tags.
However, with this gameplay, we need to design carefully to solve issues like malicious player detection,
game incentivization, disaster level calculation, etc.
In the subsequent chapters, we will discuss the system functionalities first by presenting the system mockups, 
then gives the comprehensive design of the system backend models for malicious player detection
and disaster level calculation as well. 

