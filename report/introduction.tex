\section{Introduction}

In this chapter, we will give an introduction about the backgrounds of UNICEF and the related technics 
used in the project. The simple definition of HC system and GWAPs will be given. 
The purpose of our HC system and the contribution of human beings will also be talked.

\subsection{Related Works}

\subsubsection{UNICEF}
The United Nations Children's Fund\cite{unicef1994state} is a United Nations programme headquartered
in New York City that provides humanitarian and developmental assistance to 
children and mothers in developing countries\cite{wiki:UNICEF}.
It works in 190 countries and territories to protect the rights of every child. 
UNICEF has spent 70 years working to improve the lives of children and their families. 
Defending children's rights throughout their lives requires a global presence, 
aiming to produce results and understand their effects. 
In Syrian, the UNICEF works on providing and transporting critical medicine, 
aid and supplies to the refugees living in the war areas. The challenges UNICEF meet is that 
there are many hard-to-reach (HTR) and besieged (BSG) areas and the supplies are 
very hard to be delivered to these zones if the UNICEF have no idea about 
the real time war situation and the disaster level. It will cost too much for the UNICEF 
which is just a Non-profit organization if they entirely hire employees to 
collect the data of war situation. 
Our work is to design and develop a Human Computation system by GWAPS\cite{lafourcade2015games}.

\subsubsection{HC System and GWAPs}
Human Computation system is a paradigm for utilizing the human processing power to solve problems that 
computers cannot yet solve\cite{quinn2011human}. 
It is the system of computers and large numbers of humans that work together in order to solve problems that 
could not be solved by either computers or humans alone\cite{quinn2009taxonomy}.
Our HC system is a kind of GWAPs, which uses enjoyment as the primary means of motivating participants. 
One of the challenges in any human computation system is finding a way to motivate people 
to participate\cite{quinn2011human}. 
Besides the enjoyment, we will design some interactions between users and our system to 
make the volunteer users feel honored for their contribution.

\subsection{Purpose of the System}
The users are required to select a \textbf{Region Of Interests(ROI)} upon the presented satellite images 
and tag the ROI from a provided tag list or input their own tag. Anyone can directly participant 
without registration, but the system will record an ID of each user.
Computer Graphics can also be a way to detect and recognize the map images, but it will cost 
too much time and money in developing recognition algorithms and the best 
computer graphics algorithm currently can not beat the image recognition ability of human beings. 
That's the reason we design the HC system to solve the problem.

\subsection{Human contribution to the System}
The Computer Graphic techniques and Artificial Intelligence grow very fast in recent years, 
however, it is still a great problem for computers to detect and recognize images accurately and fast.
Nevertheless, it is a simple thing for human beings to do it.
The HC system for disaster monitoring encourages more Internet users to contribute information 
to solve the image tagging problem by GWAPs. 
We developed the Player Rating Model to guarantee the quality of collected information 
and some interesting feedback and interaction are designed to maintain the enjoyment of players in the game.
Users do some image tagging tasks in the game by their computing power and intelligent 
which are contributed to collect data in the map images.

